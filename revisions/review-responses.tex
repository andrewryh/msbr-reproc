%        File: revise.tex
%     Created: Wed Oct 27 02:00 PM 2018 P
% Last Change: Wed Oct 27 02:00 PM 2018 P
%

%
% Copyright 2007, 2008, 2009 Elsevier Ltd
%
% This file is part of the 'Elsarticle Bundle'.
% ---------------------------------------------
%
% It may be distributed under the conditions of the LaTeX Project Public
% License, either version 1.2 of this license or (at your option) any
% later version.  The latest version of this license is in
%    http://www.latex-project.org/lppl.txt
% and version 1.2 or later is part of all distributions of LaTeX
% version 1999/12/01 or later.
%
% The list of all files belonging to the 'Elsarticle Bundle' is
% given in the file `manifest.txt'.
%

% Template article for Elsevier's document class `elsarticle'
% with numbered style bibliographic references
% SP 2008/03/01
%
%
%
% $Id: elsarticle-template-num.tex 4 2009-10-24 08:22:58Z rishi $
%
%
%\documentclass[preprint,12pt]{elsarticle}
\documentclass[answers,11pt]{exam}

% \documentclass[preprint,review,12pt]{elsarticle}

% Use the options 1p,twocolumn; 3p; 3p,twocolumn; 5p; or 5p,twocolumn
% for a journal layout:
% \documentclass[final,1p,times]{elsarticle}
% \documentclass[final,1p,times,twocolumn]{elsarticle}
% \documentclass[final,3p,times]{elsarticle}
% \documentclass[final,3p,times,twocolumn]{elsarticle}
% \documentclass[final,5p,times]{elsarticle}
% \documentclass[final,5p,times,twocolumn]{elsarticle}

% if you use PostScript figures in your article
% use the graphics package for simple commands
% \usepackage{graphics}
% or use the graphicx package for more complicated commands
\usepackage{graphicx}
% or use the epsfig package if you prefer to use the old commands
% \usepackage{epsfig}

% The amssymb package provides various useful mathematical symbols
\usepackage{amssymb}
% The amsthm package provides extended theorem environments
% \usepackage{amsthm}
\usepackage{amsmath}

% The lineno packages adds line numbers. Start line numbering with
% \begin{linenumbers}, end it with \end{linenumbers}. Or switch it on
% for the whole article with \linenumbers after \end{frontmatter}.
\usepackage{lineno}

% I like to be in control
\usepackage{placeins}

% natbib.sty is loaded by default. However, natbib options can be
% provided with \biboptions{...} command. Following options are
% valid:

%   round  -  round parentheses are used (default)
%   square -  square brackets are used   [option]
%   curly  -  curly braces are used      {option}
%   angle  -  angle brackets are used    <option>
%   semicolon  -  multiple citations separated by semi-colon
%   colon  - same as semicolon, an earlier confusion
%   comma  -  separated by comma
%   numbers-  selects numerical citations
%   super  -  numerical citations as superscripts
%   sort   -  sorts multiple citations according to order in ref. list
%   sort&compress   -  like sort, but also compresses numerical citations
%   compress - compresses without sorting
%
% \biboptions{comma,round}

% \biboptions{}


% Katy Huff addtions
\usepackage{xspace}
\usepackage{color}

\usepackage{multirow}
\usepackage[hyphens]{url}


\usepackage[acronym,toc]{glossaries}
%\newacronym{<++>}{<++>}{<++>}
\newacronym[longplural={metric tons of heavy metal}]{MTHM}{MTHM}{metric ton of heavy metal}
\newacronym{ABM}{ABM}{agent-based modeling}
\newacronym{ACDIS}{ACDIS}{Program in Arms Control \& Domestic and International Security}
\newacronym{AHTR}{AHTR}{Advanced High Temperature Reactor}
\newacronym{ANDRA}{ANDRA}{Agence Nationale pour la gestion des D\'echets RAdioactifs, the French National Agency for Radioactive Waste Management}
\newacronym{ANL}{ANL}{Argonne National Laboratory}
\newacronym{ANS}{ANS}{American Nuclear Society}
\newacronym{API}{API}{application programming interface}
\newacronym{ARE}{ARE}{Aircraft Reactor Experiment}
\newacronym{ARFC}{ARFC}{Advanced Reactors and Fuel Cycles}
\newacronym{ASME}{ASME}{American Society of Mechanical Engineers}
\newacronym{ATWS}{ATWS}{Anticipated Transient Without Scram}
\newacronym{BDBE}{BDBE}{Beyond Design Basis Event}
\newacronym{BIDS}{BIDS}{Berkeley Institute for Data Science}
\newacronym{CAFCA}{CAFCA}{ Code for Advanced Fuel Cycles Assessment }
\newacronym{CDTN}{CDTN}{Centro de Desenvolvimento da Tecnologia Nuclear}
\newacronym{CFD}{CFD}{Computational Fluid Dynamics}
\newacronym{CEA}{CEA}{Commissariat \`a l'\'Energie Atomique et aux \'Energies Alternatives}
\newacronym{CI}{CI}{continuous integration}
\newacronym{CNEN}{CNEN}{Comiss\~{a}o Nacional de Energia Nuclear}
\newacronym{CNERG}{CNERG}{Computational Nuclear Engineering Research Group}
\newacronym{COSI}{COSI}{Commelini-Sicard}
\newacronym{COTS}{COTS}{commercial, off-the-shelf}
\newacronym{CSNF}{CSNF}{commercial spent nuclear fuel}
\newacronym{CTAH}{CTAHs}{Coiled Tube Air Heaters}
\newacronym{CUBIT}{CUBIT}{CUBIT Geometry and Mesh Generation Toolkit}
\newacronym{CURIE}{CURIE}{Centralized Used Fuel Resource for Information Exchange}
\newacronym{DAG}{DAG}{directed acyclic graph}
\newacronym{DANESS}{DANESS}{Dynamic Analysis of Nuclear Energy System Strategies}
\newacronym{DBE}{DBE}{Design Basis Event}
\newacronym{DESAE}{DESAE}{Dynamic Analysis of Nuclear Energy Systems Strategies}
\newacronym{DHS}{DHS}{Department of Homeland Security}
\newacronym{DOE}{DOE}{Department of Energy}
\newacronym{DRACS}{DRACS}{Direct Reactor Auxiliary Cooling System}
\newacronym{DRE}{DRE}{dynamic resource exchange}
\newacronym{DSNF}{DSNF}{DOE spent nuclear fuel}
\newacronym{DYMOND}{DYMOND}{Dynamic Model of Nuclear Development }
\newacronym{EBS}{EBS}{Engineered Barrier System}
\newacronym{EDF}{EDF}{Électricité de France}
\newacronym{EDZ}{EDZ}{Excavation Disturbed Zone}
\newacronym{EIA}{EIA}{U.S. Energy Information Administration}
\newacronym{EPA}{EPA}{Environmental Protection Agency}
\newacronym{EPR}{EPR}{European Pressurized Reactors}
\newacronym{EP}{EP}{Engineering Physics}
\newacronym{EU}{EU}{European Union}
\newacronym{FCO}{FCO}{Fuel Cycle Options}
\newacronym{FCT}{FCT}{Fuel Cycle Technology}
\newacronym{FEHM}{FEHM}{Finite Element Heat and Mass Transfer}
\newacronym{FEPs}{FEPs}{Features, Events, and Processes}
\newacronym{FHR}{FHR}{Fluoride-Salt-Cooled High-Temperature Reactor}
\newacronym{FLiBe}{FLiBe}{Fluoride-Lithium-Beryllium}
\newacronym{FP}{FP}{Fission Product}
\newacronym{GDSE}{GDSE}{Generic Disposal System Environment}
\newacronym{GDSM}{GDSM}{Generic Disposal System Model}
\newacronym{GENIUSv1}{GENIUSv1}{Global Evaluation of Nuclear Infrastructure Utilization Scenarios, Version 1}
\newacronym{GENIUSv2}{GENIUSv2}{Global Evaluation of Nuclear Infrastructure Utilization Scenarios, Version 2}
\newacronym{GENIUS}{GENIUS}{Global Evaluation of Nuclear Infrastructure Utilization Scenarios}
\newacronym{GPAM}{GPAM}{Generic Performance Assessment Model}
\newacronym{GRSAC}{GRSAC}{Graphite Reactor Severe Accident Code}
\newacronym{GUI}{GUI}{graphical user interface}
\newacronym{HLW}{HLW}{high level waste}
\newacronym{HPC}{HPC}{high-performance computing}
\newacronym{HTC}{HTC}{high-throughput computing}
\newacronym{HTGR}{HTGR}{High Temperature Gas-Cooled Reactor}
\newacronym{IAEA}{IAEA}{International Atomic Energy Agency}
\newacronym{IEMA}{IEMA}{Illinois Emergency Mangament Agency}
\newacronym{IHLRWM}{IHLRWM}{International High Level Radioactive Waste Management}
\newacronym{INL}{INL}{Idaho National Laboratory}
\newacronym{IPRR1}{IRP-R1}{Instituto de Pesquisas Radioativas Reator 1}
\newacronym{IRP}{IRP}{Integrated Research Project}
\newacronym{ISFSI}{ISFSI}{Independent Spent Fuel Storage Installation}
\newacronym{ISRG}{ISRG}{Independent Student Research Group}
\newacronym{JFNK}{JFNK}{Jacobian-Free Newton Krylov}
\newacronym{LANL}{LANL}{Los Alamos National Laboratory}
\newacronym{LBNL}{LBNL}{Lawrence Berkeley National Laboratory}
\newacronym{LCOE}{LCOE}{levelized cost of electricity}
\newacronym{LDRD}{LDRD}{laboratory directed research and development}
\newacronym{LFR}{LFR}{Lead-Cooled Fast Reactor}
\newacronym{LLNL}{LLNL}{Lawrence Livermore National Laboratory}
\newacronym{LMFBR}{LMFBR}{Liquid Metal Fast Breeder Reactor}
\newacronym{LOFC}{LOFC}{Loss of Forced Cooling}
\newacronym{LOHS}{LOHS}{Loss of Heat Sink}
\newacronym{LOLA}{LOLA}{Loss of Large Area}
\newacronym{LP}{LP}{linear program}
\newacronym{LWR}{LWR}{Light Water Reactor}
\newacronym{MAGNOX}{MAGNOX}{Magnesium Alloy Graphie Moderated Gas Cooled Uranium Oxide Reactor}
\newacronym{MA}{MA}{minor actinide}
\newacronym{MCNP}{MCNP}{Monte Carlo N-Particle code}
\newacronym{MILP}{MILP}{mixed-integer linear program}
\newacronym{MIT}{MIT}{the Massachusetts Institute of Technology}
\newacronym{MOAB}{MOAB}{Mesh-Oriented datABase}
\newacronym{MOOSE}{MOOSE}{Multiphysics Object-Oriented Simulation Environment}
\newacronym{MOSART}{MOSART}{Molten Salt Actinide Recycler and Transmuter}
\newacronym{MOX}{MOX}{mixed oxide}
\newacronym{MPI}{MPI}{Message Passing Interface}
\newacronym{MSBR}{MSBR}{Molten Salt Breeder Reactor}
\newacronym{MSFR}{MSFR}{Molten Salt Fast Reactor}
\newacronym{MSRE}{MSRE}{Molten Salt Reactor Experiment}
\newacronym{MSR}{MSR}{Molten Salt Reactor}
\newacronym{NAGRA}{NAGRA}{National Cooperative for the Disposal of Radioactive Waste}
\newacronym{NEAMS}{NEAMS}{Nuclear Engineering Advanced Modeling and Simulation}
\newacronym{NEUP}{NEUP}{Nuclear Energy University Programs}
\newacronym{NFCSim}{NFCSim}{Nuclear Fuel Cycle Simulator}
\newacronym{NGNP}{NGNP}{Next Generation Nuclear Plant}
\newacronym{NMWPC}{NMWPC}{Nuclear MW Per Capita}
\newacronym{NNSA}{NNSA}{National Nuclear Security Administration}
\newacronym{NPP}{NPP}{Nuclear Power Plant}
\newacronym{NPRE}{NPRE}{Department of Nuclear, Plasma, and Radiological Engineering}
\newacronym{NQA1}{NQA-1}{Nuclear Quality Assurance - 1}
\newacronym{NRC}{NRC}{Nuclear Regulatory Commission}
\newacronym{NSF}{NSF}{National Science Foundation}
\newacronym{NSSC}{NSSC}{Nuclear Science and Security Consortium}
\newacronym{NUWASTE}{NUWASTE}{Nuclear Waste Assessment System for Technical Evaluation}
\newacronym{NWF}{NWF}{Nuclear Waste Fund}
\newacronym{NWTRB}{NWTRB}{Nuclear Waste Technical Review Board}
\newacronym{OCRWM}{OCRWM}{Office of Civilian Radioactive Waste Management}
\newacronym{ORION}{ORION}{ORION}
\newacronym{ORNL}{ORNL}{Oak Ridge National Laboratory}
\newacronym{PARCS}{PARCS}{Purdue Advanced Reactor Core Simulator}
\newacronym{PBAHTR}{PB-AHTR}{Pebble Bed Advanced High Temperature Reactor}
\newacronym{PBFHR}{PB-FHR}{Pebble-Bed Fluoride-Salt-Cooled High-Temperature Reactor}
\newacronym{PEI}{PEI}{Peak Environmental Impact}
\newacronym{PH}{PRONGHORN}{PRONGHORN}
\newacronym{PRIS}{PRIS}{Power Reactor Information System}
\newacronym{PRKE}{PRKE}{Point Reactor Kinetics Equations}
\newacronym{PSPG}{PSPG}{Pressure-Stabilizing/Petrov-Galerkin}
\newacronym{PWAR}{PWAR}{Pratt and Whitney Aircraft Reactor}
\newacronym{PWR}{PWR}{Pressurized Water Reactor}
\newacronym{PyNE}{PyNE}{Python toolkit for Nuclear Engineering}
\newacronym{PyRK}{PyRK}{Python for Reactor Kinetics}
\newacronym{QA}{QA}{quality assurance}
\newacronym{RDD}{RD\&D}{Research Development and Demonstration}
\newacronym{RD}{R\&D}{Research and Development}
\newacronym{REE}{REE}{rare earth element}
\newacronym{RELAP}{RELAP}{Reactor Excursion and Leak Analysis Program}
\newacronym{RIA}{RIA}{Reactivity Insertion Accident}
\newacronym{RIF}{RIF}{Region-Institution-Facility}
\newacronym{SFR}{SFR}{Sodium-Cooled Fast Reactor}
\newacronym{SINDAG}{SINDA{\textbackslash}G}{Systems Improved Numerical Differencing Analyzer $\backslash$ Gaski}
\newacronym{SKB}{SKB}{Svensk K\"{a}rnbr\"{a}nslehantering AB}
\newacronym{SNF}{SNF}{spent nuclear fuel}
\newacronym{SNL}{SNL}{Sandia National Laboratory}
\newacronym{STC}{STC}{specific temperature change}
\newacronym{SUPG}{SUPG}{Streamline-Upwind/Petrov-Galerkin}
\newacronym{SWF}{SWF}{Separations and Waste Forms}
\newacronym{SWU}{SWU}{Separative Work Unit}
\newacronym{TRIGA}{TRIGA}{Training Research Isotope General Atomic}
\newacronym{TRISO}{TRISO}{Tristructural Isotropic}
\newacronym{TSM}{TSM}{Total System Model}
\newacronym{TSPA}{TSPA}{Total System Performance Assessment for the Yucca Mountain License Application}
\newacronym{ThOX}{ThOX}{thorium oxide}
\newacronym{UFD}{UFD}{Used Fuel Disposition}
\newacronym{UML}{UML}{Unified Modeling Language}
\newacronym{UOX}{UOX}{uranium oxide}
\newacronym{UQ}{UQ}{uncertainty quantification}
\newacronym{US}{US}{United States}
\newacronym{UW}{UW}{University of Wisconsin}
\newacronym{VISION}{VISION}{the Verifiable Fuel Cycle Simulation Model}
\newacronym{VVER}{VVER}{Voda-Vodyanoi Energetichesky Reaktor (Russian Pressurized Water Reactor)}
\newacronym{VV}{V\&V}{verification and validation}
\newacronym{WIPP}{WIPP}{Waste Isolation Pilot Plant}
\newacronym{YMR}{YMR}{Yucca Mountain Repository Site}


\makeglossaries

%\journal{Annals of Nuclear Energy}

\begin{document}

%\begin{frontmatter}

% Title, authors and addresses

% use the tnoteref command within \title for footnotes;
% use the tnotetext command for the associated footnote;
% use the fnref command within \author or \address for footnotes;
% use the fntext command for the associated footnote;
% use the corref command within \author for corresponding author footnotes;
% use the cortext command for the associated footnote;
% use the ead command for the email address,
% and the form \ead[url] for the home page:
%
% \title{Title\tnoteref{label1}}
% \tnotetext[label1]{}
% \author{Name\corref{cor1}\fnref{label2}}
% \ead{email address}
% \ead[url]{home page}
% \fntext[label2]{}
% \cortext[cor1]{}
% \address{Address\fnref{label3}}
% \fntext[label3]{}

\title{Modeling and Simulation of Online Reprocessing in the Thorium-Fueled 
        Molten Salt Breeder Reactor\\
\large Response to Review Comments}
\author{Andrei Rykhlevskii, Jin Whan Bae, Kathryn D. Huff}

% use optional labels to link authors explicitly to addresses:
% \author[label1,label2]{<author name>}
% \address[label1]{<address>}
% \address[label2]{<address>}


%\author[uiuc]{Kathryn Huff}
%        \ead{kdhuff@illinois.edu}
%  \address[uiuc]{Department of Nuclear, Plasma, and Radiological Engineering,
%        118 Talbot Laboratory, MC 234, Universicy of Illinois at
%        Urbana-Champaign, Urbana, IL 61801}
%
% \end{frontmatter}
\maketitle
\section*{Review General Response}
We would like to thank the reviewers for their detailed assessment of
this paper. Your comments have resulted in changes which certainly improved the 
paper.


\begin{questions}
        \section*{Reviewer 1}
        %---------------------------------------------------------------------

        \question  Abstract: Given the recent announcement about TransAtomic 
        Power, you may want to remove them from your list?
        \begin{solution}
        		Thank you for your kind review. Current manuscript was submitted before 
		        announcement that TransAtomic is ceasing operations. It is removed from 
		        the list of MSR startups.
        \end{solution}

        %---------------------------------------------------------------------
        \question  Table 1, page 3: SCALE/TRITON is fast as well as thermal 
        reactor capable.  
        \begin{solution}
		        Thank you for the recommendation. In Table 1 ``thermal'' has been changed 
		        to ``thermal/fast''.
        \end{solution}

        %---------------------------------------------------------------------
        \question  Page 4, Line 48: Should read ``...and refill using a single 
        or multiple unit cell ....''
        \begin{solution}
        		That statement has been modified as requested.
        \end{solution}

        %---------------------------------------------------------------------
        \question  Page 4, Line 55. Also worth noting that the latest SCALE 
        release will have the same functionality using continuous removal (B. 
        R. Betzler, J. J. Powers, N. R. Brown, and B. T. Rearden, ``Molten Salt 
        Reactor Neutronics Tools in SCALE,'' Proc. M\&C 2017 - International 
        Conference on Mathematics \& Computational Methods Applied to Nuclear 
        Science and Engineering, Jeju, Korea, Apr. 16-20 (2017).)
        \begin{solution}
		        Thank you for the update. This sentence has been added:
        
		        The latest SCALE release will also have the same functionality using 
        		truly continuous removals \cite{betzler_implementation_2017}.
        \end{solution}

        %---------------------------------------------------------------------
        \question  Page 8, Fig 2: Appears the image is cut off at the top?
        \begin{solution}
        		The image has been replotted.
        \end{solution}

        %---------------------------------------------------------------------
        \question  Page 13, line 209: The description of the Pa removal, 
        although correct, isn't quite fully correct. The reason the Pa is 
        removed from the core and hence flux is to then enable the Pa to decay 
        to U233. If it was left in the core, it would transmute further and 
        hence not be able to produce the U233 that is necessary for this 
        breeding cycle to work.
        \begin{solution}
        		Thanks for catching this. Following passage has been added:
        
		        Protactinium presents a challenge, since it has a large absorption cross 
				section in the thermal energy spectrum. Moreover, $^{233}$Pa left in the core
				 would produce $^{234}$Pa and $^{234}$U, which both are not useful as fuel, 
				and smaller amount of $^{233}$Pa would decays into the fissile $^{233}$U.
				Accordingly, $^{233}$Pa is continuously 
				removed from the fuel salt into a protactinium decay tank to allow $^{233}$Pa 
				to decay to $^{233}$U without negative impact on neutronics.
        \end{solution}

        %---------------------------------------------------------------------
        \question  Table 3: ``cycle time'' is not defined in the paper. Please 
        add.
        \begin{solution}
		        The ``cycle time'' definition has been added in a first appearance in text.
        \end{solution}

        %---------------------------------------------------------------------
        \question  Page 14, line 224. The 3 day time step as the ``optimum'' for 
        Th fuel cycles in an MSR was first described and concluded by Powers et 
        al. Please add a reference to their initial work.
        \begin{solution}
        		The reference to \cite{powers_new_2013} has been added.
        \end{solution}

        %---------------------------------------------------------------------
        \question  Page 14, line 234 onwards: Doesn't SERPENT already have an 
        MSR removal capability? If so, what is different about using SaltProc 
        with SERPENT?
        \begin{solution}
                It does. We tried use it before but these capabilities needs
                 to be verified. Additional text to clarify this point has 	
                been add to solution for question 21.
        \end{solution}

        %---------------------------------------------------------------------
        \question  Page 18, Figure 7: The figure is hard to interpret or see 
        clearly what is going on. Could an additional figure or a zoomed in 
        portion be added to show what the swing in k is over a much shorter 
        time interval? k seems to be swinging dramatically but over what time 
        period and how would this be controlled in reality? The graph almost 
        suggests that the core is unstable??
        \begin{solution}
                Zoomed portion for 150 EFPD interval has been added. We also 
                added notes on a plot to explain swing in multiplication factor.
        \end{solution}

        %---------------------------------------------------------------------
        \question  Page 18, line 327: Are those elements removed every 3435 
        days, or is it that the entire salt core is discharged?
        \begin{solution}
		        100 \% of those elements atoms removed every 3435 days. Full salt discard
		         as it mentioned in Table 3 has not been implemented. More detailed 
		         explanation of this has been added:
        
		        Additionally, the presence of rubidium, strontium, cesium, and barium in
		         the core are disadvantageous to reactor physics. In fact, SaltProc fully 
		         removes all these elements every 3435 days (not a small mass fraction 
		         every 3 days) which causes the multiplication factor to jump by 
		        approximately 450 pcm, and limits using the batch approach for online 
		        reprocessing simulations. In future versions of SaltProc this drawback 
		        will be eliminated by removing elements with longer cycle times (seminoble 
		        metals, volatile fluorides, Rb, Sr, Cs, Ba, Eu) using different approach. 
		        Only mass fraction (calculated separately for each reprocessing group) 
		        will be removed every depletion step (e.g. 3 days) instead of removing 
		        100\% of element atoms after cycle time. 
        \end{solution}

        %---------------------------------------------------------------------
        \question  Page 19, Figure 8 (and same for Fig 9): y-axis in grams/kgs 
        or mass units would be better for the reader.
        \begin{solution}
                Thank you for the recommendation. Atom density was chosen for publication-to-						publication comparison (e.g. Park \emph{et al.} and Betzler \emph{et al.}
                \cite{park_whole_2015, betzler_molten_2017}). Although mass would certainly 
                be more understandable for the reader and will be added in a future releases.
        \end{solution}

        %---------------------------------------------------------------------
        \question  Page 20, Fig 9: What are the wiggles and dips , especially 
        seen for Np235?
        \begin{solution}
		        Explanation of this phenomena has been added as follows:
        
		         Small dips in neptunium and plutonium number density every 16 years are 
		         caused by removing $^{237}$Np, $^{242}$Pu (included in Processing group 
		         ``Higher nuclides'', see Table 3) which decays
        		  into $^{235}$Np, $^{239}$Pu, respectively. 
        \end{solution}

        %---------------------------------------------------------------------
        \question  Page 20, line 351: It is more than just the Pu isotopes that 
        makes the spectrum harder? What about the other MAs etc?
        \begin{solution}
        		Thank you for the excellent point. The corrected sentence reads thus: 
        
		        The neutron energy spectrum at equilibrium is harder than at startup due 
                to plutonium and other strong absorbers accumulation in the core during 
		        reactor operation.  
        \end{solution}

        %---------------------------------------------------------------------
        \question  Fig 12: units on y-axis?
        \begin{solution}
                Thanks for catching this. Units $\frac{n}{cm^2 s}$ has been added.
        \end{solution}

        %---------------------------------------------------------------------
        \question  Page 24, line 389: Should that be ``233U'' and not ``233Th''?
        \begin{solution}
                Yes, we meant $^{233}$U production. The typo has been fixed.
        \end{solution}

        %---------------------------------------------------------------------
        \question  Table 5: Please provide some comments on the uncertainties - 
        where do they come from? Also, the ``reference'' results need to state 
        whether ``initial'' or ``equilibrium''
        \begin{solution}
                In the Table 5 was added information that reference column contains 
                data for initial fuel salt composition. Details about uncertainties
                 have been added:
                
                Uncertainty for each temperature coefficient was obtain by 
                propagating statistical error of effective multiplication factor 
                calculation provided by SERPENT2 and also appears in Table 5. 
                Other sources of uncertainty, such as cross section libraries 
				uncertainty, error in salt and graphite density correlations, 
				are not treated here.
        \end{solution}

        %---------------------------------------------------------------------
        \question  Page 26, line 425: ``Relatively large'' compared with what? 
        Perhaps results for an LWR case would be a good basis for comparison?
        \begin{solution}
                Sentence has been extended as follows:
                
                ...the total temperature coefficient of reactivity is relatively large and
                 negative during reactor operation (comparing with conventional PWR which 							temperature coefficient about -1.71 pcm/$^\circ$F $\approx$ -3.08 pcm/K 							\cite{forget_integral_2018}), despite positive MTC, and affords excellent 
                 reactor stability and control.
        \end{solution}

        %---------------------------------------------------------------------
        \question  Page 27, section 3.8: It needs to be made more clear that 
        these results were calculated, and that they are taken from the code 
        output.
        \begin{solution}
                This has now been clarified in the text:
                
                Table 7 summarizes the six factors for both initial and 
				equilibrium fuel salt composition. These factors and their statistical 								uncertainties have been calculated using SERPENT2 code for initial fuel salt 
				composition (see Table 2) and for equilibrium salt composition which was 
				obtained with SaltProc. 
        \end{solution}

        %---------------------------------------------------------------------
        \question  Page 29, Figure 15: Similar comment to above regarding Fig 7 
        - the results are difficult to see and interpret with such notable 
        swings.
        \begin{solution}
                Zoomed portion for 150 EFPD interval and clarifying 
                notes has been added.
        \end{solution}

        %---------------------------------------------------------------------




        \section*{Reviewer 2}

        %---------------------------------------------------------------------

        \question The most critical point of the work is that the built-in 
        capabilities for online reprocessing of Serpent 2 have not been used. 
        Their use is mentioned in the future work, but it is not clear why 
        these capabilities have not been used in the current work. To the 
        author's knowledge, they have been available in Serpent 2 since quite a 
        while. The authors should clarify this point at the beginning of the 
        paper, and not only in the ``Future work'' section. Even though the 
        technical work was done without using these capabilities, they should 
        highlight what  SaltProc adds to the built-in Serpent capabilities, and 
        they should at least try to extrapolate on the potential advantages of 
        combining  SaltProc and Serpent capabilities. Based on this, they 
        should slightly restructure the paper in order to prove the claimed 
        advantages over Serpent 2.
        \begin{solution}
                Thank you for your kind review. We tried to use these capabilities 
                \cite{rykhlevskii_online_2017} but have had number of issues which 
                are hard to resolve due to lack of documentation and publications 
                involving this feature. Following paragraph has been added to 
                clarify why these capabilities have not been used in current work:

                We employed this extended SERPENT2 for a simplified unit-cell 
                geometry of thermal spectrum thorium-fueled MSBR and have had 
                following problems: (1) lack of documentation describing how 
                to use this built-in SERPENT2 online reprocessing capabilities
                \footnote{ Some challenges in no particular order: mass 
                conservation is hard to achieve; not clear when use mflow card 
                type 0, 1 or 2; difference between CRAM and TTA results; etc.} 
				(Discussion forum for SERPENT users is only useful 
				source of information at the moment); (2) reactivity control 
				module described in Aufiero \emph{et al.} is not available 
				in the latest SERPENT 2.1.30 release; (3) infinite multiplication 
				factor behavior for simplified unit-cell model obtained using 
				SERPENT2 built-in capabilities \cite{rykhlevskii_online_2017} 
				does not match with exist MCNP6/Python-script results for the 
				similar model by Jeong and Park\footnote{ In our study 
				k$_{\infty}$ drops from 1.05 to 1.005 during a 1200 days of 
				depletion simulation while in Jeong and Park work this parameter 
				decreasing slowly from 1.065 to 1.05 for the 
				similar time-frame.} \cite{jeong_equilibrium_2016}; (4) only few 
				publication \cite{aufiero_extended_2013, ashraf_nuclear_2018} 
				using these capabilities are available which is indicative of 
				lack of reproducibility. Nevertheless, truly continuous 
				online reprocessing capabilities of SERPENT2 is desirable feature 
				for our future research. However, these capabilities should be 
				carefully verified against ChemTriton/SCALE or proposed 
				SaltProc/SERPENT2 package, and could be employed 
				for removal of fission products with shorter residence time 
				(e.g., Xe, Kr) which have a strong negative impact on core 
				lifetime and breeding efficiency which we intent to do in 
				the nearest future.
        \end{solution}

        %---------------------------------------------------------------------

        \question Considering the scope of the journal, mentioning the names of 
        companies and start-up is not appropriate. I would suggest removing 
        them.
        \begin{solution}
                The names of companies have been removed.
        \end{solution}

        %---------------------------------------------------------------------
        
        \question The sentence ``Immediate advantages over traditional, solid-fueled, 
        reactors include near-atmospheric pressure in the 15 primary loop, 
        relatively high coolant temperature, outstanding neutron economy, 
        improved safety parameters,'' may suggest improved neutron economy vs 
        solid-fuel fast reactors. This is rarely the case, especially vs 
        Pu-based SFRs. I would suggest reformulating.
        \begin{solution}
                Thank you for the exceptional recommendation. The sentence has been 
                changed as follows:
                
                ``Immediate advantages over traditional light water reactors 
                include\dots''
        \end{solution}

        %---------------------------------------------------------------------

        \question The sentence ``With regard to the nuclear fuel cycle, the 
        thorium cycle produces a reduced quantity of plutonium and minor 
        actinides (MAs) compared to the traditional uranium fuel cycle'' is 
        correct, but the fact that this is an advantage is questionable. The 
        pros\&cons of thorium cycle have been long debated and there is no 
        consensus on its advantage in terms of exposure of workers, exposure of 
        public, geological repository, etc. I would suggest removing the 
        sentence.
        \begin{solution}
                Thanks for your kind comment. The sentence has been removed.
        \end{solution}

        %---------------------------------------------------------------------


        \question ``Methods listed in references [14, 17, 24, 25, 28, 29, 30] 
        as well as the current work also employ a batch-wise approach''. As a 
        matter of fact, the work in [14]  allows for continuous reprocessing 
        via introduction of ``reprocessing'' time constants. The work from  
        Aufiero (mentioned in the following paragraph) actually used the 
        methodology previously developed in [14]  for verification purposes.
        \begin{solution}
                Thank you for the information. [14] has been removed from the 
                sentence, and following paragraph has been modified to read:
                
                Accounting for continuous removal or addition presents a greater 
                challenge since it requires adding a term to the Bateman equations. 
                Fiorina \emph{et al.} simulated MSFR depletion with continuous
                 fuel salt reprocessing via introducing ``reprocessing'' time 
                constants into ERANOS transport code \cite{fiorina_investigation_2013}. 
                Aufiero \emph{et al.} improved SERPENT2 using similar methodology by 
                explicitly introducing continuous reprocessing in the system of Bateman 
                equations by adding effective decay and transmutation terms for the 
                different nuclides \cite{aufiero_extended_2013}. 
        \end{solution}

        %---------------------------------------------------------------------

        \question Table 3. It is not clear what ``effective cycle times'' are. 
        Please clarify.
        \begin{solution}
                The ``cycle time'' definition has been added in a first appearance in text.
        \end{solution}

        %---------------------------------------------------------------------

        \question The removal of fission products is made batch wise in the 
        described algorithms. However it is not clear how the fission products 
        with the longest ``effective cycle times'' are removed. Part of them at 
        every batch? Or all of them at the end of the  ``effective cycle 
        times''.  Please clarify. And please clarify the relation between 
        ``effective cycle times'', batches and the average time spent by a 
        fission product in the reactor.
        \begin{solution}
                All of them at the end of the cycle time, and we agree that it
                 was not best solution. To clarify this following paragraph has 
                been added:
                 
				Current version of SaltProc only allows separate out 100\% of 
				specific elements or group of elements (e.g. Processing Groups 
				as described in Table 3 at the end
				 of the specific cycle time. This approach works well for 
				fast-removing elements (gases, noble metals, protactinium) which 
				should be removed each depletion step. Unfortunately, for the 
				elements with longer cycle time (i.e. rare earths should be removed 
				every 50 days) this simplified approach leads to oscillatory 
				behavior of all major parameters. In future releases of SaltProc
				 this drawback will be eliminated by removing elements with longer 
				cycle times using different method: only mass fraction (calculated 
				separately for each reprocessing group) will be removed each 
				depletion step (e.g. 3 days).	
        \end{solution}

        %---------------------------------------------------------------------

        \question The oscillatory behavior shown in Fig. 7 on the time scale of 
        months/years is hard to explain. Is this because the fission product 
        with longer residence time are batch-wise removed at their ``effective 
        cycle time''? In case, why not to remove part of them at every depletion 
        step?  
        \begin{solution}
                Yes, the oscillation happened because the fission products 
        		with longer residence time are removed at the end of cycle time. We are 
                definitely will take your advice and improve the code in future 
                releases. Following text has been added to clarify this issue:
                
                In fact, SaltProc fully removes all these elements every 3435 days 
                (not a small mass fraction every 3 days) which causes the 
                multiplication factor to jump by approximately 450 pcm, and limits 
                using the batch approach for online reprocessing simulations. In 
                future versions of SaltProc this drawback will be eliminated by 
                removing elements with longer residence time (seminoble metals, 
                volatile fluorides, Rb, Sr, Cs, Ba, Eu) using different approach. 
                Only mass fraction (calculated separately for each reprocessing 
                group) will be removed every depletion step (e.g. 3 days) instead 
                of removing 100\% of element atoms after cycle time.
        \end{solution}

        %---------------------------------------------------------------------

        \question Can the proposed tool adjust reactivity?
        \begin{solution}
                No, and we do not plan to add this capability.
        \end{solution}

        %---------------------------------------------------------------------

        \question ``The main physical principle underlying the reactor 
        temperature feedback is an expansion of material that is heated''. This 
        sentence would deserve some support data. Can you please calculate 
        separately the effect of temperature (Doppler in fuel and spectral 
        shift in graphite) and density?
        \begin{solution}
                Thank you for the excellent recommendation. Effects of temperature
                 and density have been calculated separately and added in Table 5.
                 Moreover, during these simulations we have discovered mistake in 
                 fuel salt and graphite density correlations and completely 
                 recalculated temperature coefficients of reactivity with lower 
                 statistical error. Now initial total temperature coefficient is 
                 closer to the reference and statistical uncertainty was reduced 
                 from 0.046 to 0.038 pcm/K.
        \end{solution}

        %---------------------------------------------------------------------

        \question How uncertainties have been calculated in Table 5? Is this 
        just statistical uncertainty from Serpent calculations? Please clarify
        \begin{solution}
                Yes, uncertainties were determined from statistical error 
                from SERPENT output. Following passage has been added to clarify
                 this point:
                 
                 Uncertainty for each temperature coefficient was obtain by 
                 propagating statistical error of effective multiplication factor 
                 calculation provided by SERPENT2 and also appears in Table 5. 
                 Other sources of uncertainty, such as cross section libraries 
				 uncertainty, error in salt and graphite density correlations, 
				 are not treated here.
        \end{solution}

        %---------------------------------------------------------------------

        \question For calculating the coefficients in table 5, are you only 
        changing densities, or also dimensions? Please clarify and provide a 
        justification.  
        \begin{solution}
        		We have changed both densities and dimensions. This passage 
        		has been modified to read:
        		
                A new geometry input for SERPENT2, which takes into account 
                displacement of graphite surfaces, was created based on this information. 
                For the displacements calculation it was assumed that the interface 
                between graphite reflector and vessel did not move, and the vessel 
                temperature did not change. This is the most reasonable assumption for 
				the short-term reactivity effects because inlet salt is cooling graphite 
				reflector and inner surface of the vessel.
        \end{solution}

        %---------------------------------------------------------------------

        \question ``The fuel temperature coefficient (FTC) is negative for both 
        initial and equilibrium fuel compositions due to thermal Doppler 
        broadening of the resonance capture cross sections in the thorium.'' 
        What is the effect of density?
        \begin{solution}
                This passage has been modified to illuminate the effect of density:
                
				The fuel temperature coefficient (FTC) is negative for both initial and 
				equilibrium fuel compositions due to thermal Doppler broadening of the 
				resonance capture cross sections in the thorium. Small positive effect 
				of fuel density on reactivity is increasing from +1.21 pcm/K for the 
				reactor startup to +1.66 pcm/K for equilibrium fuel composition which 
				has negative effect on FTC magnitude during the reactor operation. 
        \end{solution}

        %---------------------------------------------------------------------

        \question ``This thorium consumption rate is in good agreement with a 
        recent online reprocessing study by ORNL [29].'' Please notice that in a 
        reactor with only Th as feed, and near equilibrium, the Th consumption 
        rate is  exclusively determined by the reactor power and by the energy 
        released per fission.  
        \begin{solution}
                Thank you for the recommendation. Following sentence has been 
                added:
                
                It must be noted that for the reactor with only thorium feed, 
                at near equilibrium state, the thorium consumption rate is 
				determined by the reactor power, the energy released per fission, 
				and neutron energy spectrum.
        \end{solution}

        %---------------------------------------------------------------------

        \question Are you considering the effect of the gradual poisoning of 
        the graphite with fission products? If not, it would be worthwhile 
        briefly discussing its effect.
        \begin{solution}
                This passage has been added:
				
				$^{135}$Xe is a strong poison to the reactor, and some 
				fraction of this gas is absorbed by graphite during MSBR
				operation. ORNL calculations shown that for unsealed commercial 
				graphite with helium permeability 10$^{-5}$ cm$^2$/s the 
				calculated poison fraction is less than 2\% \cite{robertson_conceptual_1971}. 
				This parameter can be improved by using experimental graphites 
				or by applying sealing technology. The effect of the gradual 
				poisoning of the core graphite with xenon is not treated here.
        \end{solution}

        %---------------------------------------------------------------------

        \question In the manuscript it is not always clear when the authors 
        refer to numerical approximations of physical situations. For instance, 
        the authors write ``Figure 16 demonstrates that batch-wise removal of 
        strong absorbers every 3 days did not necessarily leads to fluctuation 
        in results but rare earth elements 480 removal every 50 days causes an 
        approximately 600 pcm jump in reactivity.'' These 600 pcm are an effect 
        of a numerical approximation, but the way things are presented can be 
        confusing to the reader. Please try to explicitly separate physical and 
        numerical effects. And try to related numerical effects to physical 
        effects. For instance, how does these numerical ``jumps'' affect the 
        results?  In this sense, why the batch-wise removal of strong absorbers 
        every 3 days  was not done? And why a fraction of rare earths  is not 
        removed every three days?
        \begin{solution}
                <++>
        \end{solution}

        %---------------------------------------------------------------------



        \question ``The current work results show that a true equilibrium 
        composition cannot exist but balance between strong absorber 
        accumulation and new fissile material production can be achieved to 
        keep the reactor critical.'' Not clear. Do you mean that the equilibrium 
        composition cannot be achieved in a lifetime of the reactor? Please 
        clarify
        \begin{solution}
                This statement has been removed completely.
        \end{solution}

        %---------------------------------------------------------------------






        \section*{Reviewer 3}

        %--------------------------------------------------------------------

        \question  What are the main differences of this work with the previous 
        works, especially with the work published by Park at al. in ``Whole core 
        analysis of molten salt breeder reactor with online fuel reprocessing''?
        \begin{solution}
                Thank you for your question. The new paragraph has been added:
                
                The works described in \cite{park_whole_2015} and 
                \cite{jeong_equilibrium_2016} are most similar to the work 
                presented in this paper. However, few major differences worse to 
                be mentioned: (1) Park \emph{et al.} employed MCNP6 for depletion 
				simulations while SERPENT2 was used in this work (SERPENT2 seems 
				to be much faster in massively parallel simulations involving more 
				than 1000 cores); (2) full-core reactor model herein is more 
				detailed and has no major approximation comparing with Park \emph{et al.}
				(detailed full-core models comparison might be found elsewhere
				\cite{rykhlevskii_full-core_2017}); (3) Park \emph{et al.} and 
				Jeong \emph{et al.} both considered volatile gases, noble metals 
				removal and $^{233}$Pa separation while in the current work we are 
				implemented detailed reprocessing scheme described in conceptual 
				MSBR design \cite{robertson_conceptual_1971}; (4) $^{232}$Th 
				neutron capture reaction rate has been investigated to prove 
				advantages of two-region core design; (5) the effect of removing 
				fission product removal from fuel salt has been studied. 
        \end{solution}

        %--------------------------------------------------------------------
        \question  How did the authors verify the coupled SaltProc and Serpent 
        code system?  
        \begin{solution}
                We have compared few parameters (multiplication factor, Th refill 
                rate, neutron energy spectrum) with Betzler \emph{et al.} 											\cite{betzler_molten_2017} and mentioned it in the Result section. 
                We also compared neutron energy spectrum and temperature coefficients
                 for equilibrium composition with Park \emph{et al.} \cite{park_whole_2015}.
                In a future SaltProc release suite of unit tests will be added 
                to make sure that results are consistent with our test cases.
        \end{solution}

        %--------------------------------------------------------------------
        \question  In Page 3, the title of ``Table 1'' should not only contain 
        the ``fast spectrum system''. The work published by Zhou and Yang et al. 
        with the title of ``Fuel cycle analysis of molten salt reactors based on 
        coupled neutronics and thermal-hydraulics calculations'' needs to be 
        included in Table 1 for completeness.
        \begin{solution}
                Thank you for your excellent recommendations. The table has been 
                enriched as requested.
        \end{solution}

        %--------------------------------------------------------------------
        \question  In Page 5, the following sentence needs to be explained. ``We 
        employed this extended SERPENT 2 for a simplified unit-cell geometry of 
        thermal spectrum thorium-fueled MSBR and obtained results which 
        contradict existing MSBR depletion simulations.''
        \begin{solution}
                This statement has been significantly extended (see Solution 21).
        \end{solution}

        %--------------------------------------------------------------------
        \question  In Page 10, ``SERPENT generates the problem-dependent nuclear 
        data library'', how did SERPENT generate the problem-dependent nuclear 
        data library? What kind of nuclear data library did SERPENT generate?
        \begin{solution}
                Thank you for your kind review. The statement has been removed
		        entirely and we emphasized that temperature of each material is 
		        assumed to be constant over 60 years:
		        
		        The specific temperature was fixed for each material and did 
		        not change during the reactor operation. 
        \end{solution}

        %--------------------------------------------------------------------
        \question  In Page 14, ``depletioncalculations'' should be ``depletion 
        calculations''.
        \begin{solution}
                Thanks for catching this, fixed.
        \end{solution}

        %--------------------------------------------------------------------
        \question  In Page 21, it looks like the neutron spectrum is not 
        normalized in Figure 10. It is recommended to normalize the neutron 
        spectrum for comparison.
        \begin{solution}
                Thank you for the comment. Neutron energy spectrum in Figure 
                10 and 11 are normalized per unit lethargy and the area under 
                the curve is normalized to 1.
        \end{solution}

        %--------------------------------------------------------------------
        \question  In Page 22, ``Figure 13 reflects the normalized power 
        distribution of the MSBR quarter core, which is the same at both the 
        initial and equilibrium states'' contradicts the following statement of 
        ``The spectral shift during reactor operation results in different power 
        fractions at startup and equilibrium''.
        \begin{solution}
                Thanks for catching this. The difference between power fraction
                 is very small that could be seen from Table 4. It is impossible
                  to see the difference in a contour plot that is why we left only 
                 equilibrium composition on Figures 13 and 14. The paragraph has 
                 been modified as follows:
                 
                 Table 4 shows the power fraction in each zone for initial and 
                 equilibrium fuel compositions. Figure 13 reflects the normalized 
                 power distribution of the MSBR quarter core for equilibrium 
                 fuel salt composition. For both the initial and equilibrium 
                 compositions, fission primarily occurs in the center of the core, 
                 namely zone I. The spectral shift during reactor operation results 
                 in slightly different power fractions at startup and equilibrium, 
                 but most of the power is still generated in zone I at equilibrium 
				 (table 4).                  
                   
        \end{solution}

        %--------------------------------------------------------------------
        \question  In Page 24, it is hard to agree with the statement that ``the 
        majority of 233Th is produced in zone II.''. How did the authors draw 
        this conclusion?
        \begin{solution}
                The statement has been removed entirely.
        \end{solution}

        %--------------------------------------------------------------------
        \question  In Page 24 and 25, why did the normalized power density 
        distribution and the 232Th neutron capture reaction rate distribution 
        share the same figure for both initial and equilibrium fuel salt 
        compositions?
        \begin{solution}
                Thanks for catching this, it is a typo. Figure 13 and 14 are plotted 
                for equilibrium composition only.            	    
        \end{solution}

        %--------------------------------------------------------------------
        \question  Too much background information was contained in the 
        abstract, which deteriorates the readability of the abstract.
        \begin{solution}
                <++>
        \end{solution}

        %--------------------------------------------------------------------
\end{questions}
\bibliographystyle{unsrt}
\bibliography{../2018-msbr-reproc}
\end{document}

%
% End of file `elsarticle-template-num.tex'.
