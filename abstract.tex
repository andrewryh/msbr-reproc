\begin{abstract}

Current interest in advanced nuclear energy and \gls{MSR} concepts has enhanced demand in building the tools to analyze these systems. This paper introduces a Python script, SaltProc, which simulates \gls{MSR} online reprocessing by modeling the changing isotopic composition of an irradiated fuel salt. SaltProc couples with the Monte Carlo code, SERPENT 2, for neutron transport and depletion calculations. SaltProc capabilities was illuminated for full-core depletion in the \gls{MSBR} which demonstrated that (1) equilibrium fuel composition could be achieved after 16 years and (2) the multiplication factor stabilizes after 6 years of operation. Fuel salt irradiation with online reprocessing causes considerable neutron energy spectrum hardening; this spectral shift has a problematic impact on safety parameters (e.g., temperature reactivity feedback, reactivity control system worth). Volatile gases and noble metals removal has positive effect on the core performance approximately 7500 pcm over 10-years timeframe; rare earth elements removal has slower rate and less impact on reactivity (5500 pcm). Finally, the average $^{232}$Th feed rate throughout 60 years of operation is 100 g/GWh$_e$ which is a good agreement with other recent research. Problematic effects of neutron energy spectrum hardening during \gls{MSBR} operation should be taken into account for neutronics, multi-physics, and fuel cycle performance analysis.

\end{abstract}

