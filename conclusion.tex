\section{Discussion and conclusions}

This work introduces the open source \gls{MSR} simulation code SaltProc. The SaltProc modeling and simulation tool expands the capability of a continuous-energy Monte Carlo Burnup calculation code SERPENT 2 for analyzing liquid-fueled \gls{MSR} operation \cite{andrei_rykhlevskii_arfc/saltproc:_2018}. Benefits of SaltProc include generic geometry modeling, multi-flow capabilities, time-dependent feed and removal rates, and the ability to specify removal efficiency. The main goal has been to demonstrate the ability of this tool to find equilibrium fuel salt composition (when the number densities of major isotopes vary less than 1\% over several years). A secondary goal has been to compare predicted operational and safety parameters (e.g., neutron energy spectrum, power and breeding distribution, temperature coefficients of reactivity) of the \gls{MSBR} at startup and equilibrium state. A tertiary goal has been to demonstrate benefits of continuous fission products removal for thermal \gls{MSR} design.

Toward these goals, a full-core high-fidelity benchmark model of the \gls{MSBR} was implemented in SERPENT 2. The purpose of the full-core model instead of simplified single-cell model \cite{betzler_molten_2017, rykhlevskii_online_2017, betzler_fuel_2018} was to precisely describe the two-region \gls{MSBR} concept design sufficiently to accurately represent breeding in the ``blanket" (outer core zone). When running depletion calculations, the most important fission products and $^{233}$Pa are removed and fertile/fissile materials are added to fuel salt every 3 days, while the removal interval for the rare earths, volatile fluorides and seminoble metals was more than month which causes effective multiplication factor fluctuation. 

\subsection{Equilibrium state search}
The results of this study indicate that the effective multiplication factor slowly decreases from 1.075 and reaches 1.02 at equilibrium after approximately 6 years of operation. At the same time, concentration of $^{233}$U, $^{232}$Th, $^{233}$Pa, $^{232}$Pa stabilizes after approximately 2500 days of operation. Particularly, $^{233}$U number density fluctuates less than 0.8\% from 16 to 20 years of operation. Consequently, the core reaches the quasi-equilibrium state after 16 years of operation. On the other hand, a wide diversity of nuclides, including fissile isotopes (e.g. $^{233}$U, $^{239}$Pu) and non-fissile strong absorbers (e.g. $^{234}$U), keep accumulating in the core. Current work results show that a true equilibrium composition cannot exist but the balance between negative effects of strong absorber accumulation and new fissile material production can be achieved to keep the reactor critical.

\subsection{Spectral shift}
Another finding to emerge from the analysis of initial and equilibrium isotopic composition is that the neutron energy spectrum is harder for the equilibrium state because significant amount of heavy fission products were accumulated in the \gls{MSBR} core. Moreover, the neutron energy spectrum in the central core region is much softer than in the outer core region due to lower moderator-to-fuel ratio in the outer zone, and this distribution remains stable during reactor operation. Finally, the epithermal or thermal spectrum is needed to effectively breed $^{233}$U from $^{232}$Th because radiative capture cross section of thorium-232 monotonically decreases from $10^{-10}$ MeV to $10^{-5}$ MeV. A harder spectrum in the outer core region tends to significantly increase resonance absorption in thorium and decrease the absorptions in fissile and structural materials. 

The spatial power distribution in the \gls{MSBR} shows that 98\% of the fission power is generated in central zone I, and neutron energy spectral shift did not cause any notable changes in a power distribution. The neutron capture reaction rate spatial distribution for fertile $^{232}$Th, corresponding to breeding in the core, confirms that most of the breeding occurs in an outer, undermoderated, region of the \gls{MSBR} core. Finally, the average $^{232}$Th refill rate throughout 60 years of operation is approximately 2.40 kg/day or 100 g/GWh$_e$.

Comparisons of the safety parameters were made for the initial fuel loading and equilibrium compositions with the SERPENT 2 Monte Carlo code. It is noted that neutron energy spectrum hardening over the fuel depletion and this spectral shift causes changes in the reactor behavior. The total temperature coefficient is large and negative at startup and equilibrium but the magnitude decreases throughout reactor operation from $-3.10$ to $-0.94$ pcm/K. The moderator temperature coefficient is positive and also decreases during fuel depletion. From reactivity control system efficiency analysis, the safety rod integral worth decreases by approximately 16.2\% over 16 years of operation, while graphite rod integral worth remains constant. Summing up, neutron energy spectrum hardening during fuel salt depletion has an undesirable impact on \gls{MSBR} stability and controllability, and should be taken into consideration in further analysis of accident transient scenarios.

\subsection{Benefits of fission products removal}
Specific chemical elements separation from molten salt is a sophisticated engineering problem that requires    extensive R\&D work (designing separation plant, bypass flow system) and has significant economical costs. Moreover, storing and handling separated radioactive material that might be in solid, liquid and gaseous form, presents an engineering challenge. On the other hand, keeping all fission products in the molten salt  also impossible because some elements are chemically antagonistic with the molten fuel salt components. In fact, some \gls{MSR} designs include removals of only few elements (e.g., volatile gases sparging). Other fission products removal is economical problem: the increase of core performance should outweigh the cost of removal.

For thermal spectrum \gls{MSBR} removal of volatile gases, noble metals, and rare earths from a fuel salt has significant core performance benefit. Moreover, immediate removal of volatile gases (e.g., xenon) and noble metals had positive effect on reactivity approximately 7500 pcm over 10-years timeframe. In contrast, the effect of relatively slower removal of rare earths elements (every 50 days cycle instead of 3 days) has less impact (5500 pcm) on the core reactivity after 10 years of operation. In sum, additional study needed to carefully estimate neutronics benefits of removing each element with understanding of chemical separation performance and costs.

\subsection{Future work}
Continued research into SaltProc-SERPENT and related topics could progress in a number of different directions. First and foremost, efforts should be made to enable optimization of reprocessing parameters (e.g. time step, feeding rate, protactinium removal rate) to achieve the best fuel utilization, breeding ratio or safety characteristics. This might be performed with a parameter sweeping outer loop which would change an input parameter by a small increment, run the simulation and analyze output to determine optimal configuration. Furthermore, the existing RAVEN optimization framework might be employed for this optimization study \cite{alfonsi_raven_2013}.

Only the batch-wise online reprocessing approach has been treated in this thesis. However, the SERPENT 2 Monte Carlo code was recently extended for continuous online fuel reprocessing simulation \cite{aufiero_extended_2013}. This extension must be verified against existing SaltProc/SERPENT or ChemTriton/SCALE packages, and could be employed for immediate removal of fission product gases (e.g., Xe, Kr) which have a strong negative impact on core lifetime and breeding efficiency. Finally, using the built-in SERPENT 2 Monte Carlo code online reprocessing \& refueling material burnup routine would significantly speed up computer-intensive full-core depletion simulations.

Lastly, an additional area to explore is the accident safety analysis which requires development a multi-physics model of the \gls{MSBR} with the coupled neutronics/thermal-hydraulics code, Moltres \cite{lindsay_introduction_2018}. The existing full-core SERPENT 2 model and equilibrium fuel material composition would be employed to generate problem-oriented nuclear data libraries for further usage in accident transient analysis. The final goal of this effort is to develop a fast-running computational model which could study the dynamic behavior of generic \glspl{MSR}, performing detailed safety analysis and design optimization for a variety of reactor concepts.
