\section{Discussion and conclusions}

This work introduces the open source \gls{MSR} simulation package SaltProc. 
SaltProc expands the capability of SERPENT 2, the continuous-energy Monte Carlo 
code to include online reprocessing in liquid-fueled \gls{MSR} operation 
\cite{andrei_rykhlevskii_arfc/saltproc:_2018}. Benefits of SaltProc include 
generic geometry modeling, multi-flow capabilities, time-dependent feed and 
removal rates, and the ability to specify removal efficiency. The main goal of 
this work has 
been to demonstrate SaltProc's capability to find the equilibrium fuel salt 
composition (where equilibrium is defined as when the number densities of major 
isotopes vary by less than 1\% over several years). A secondary goal has been to 
compare predicted operational and safety parameters (e.g., neutron energy 
spectrum, power and breeding distribution, temperature coefficients of 
reactivity) of the \gls{MSBR} at startup and equilibrium state. A tertiary goal 
has been to demonstrate benefits of continuous fission products removal for 
thermal \gls{MSR} design.

To achieve these goals, a full-core high-fidelity benchmark model of the \gls{MSBR} 
was implemented in SERPENT 2. The full-core model was used instead of the 
the simplified single-cell model \cite{betzler_molten_2017, 
rykhlevskii_online_2017, betzler_fuel_2018} to precisely describe the 
two-region \gls{MSBR} concept design sufficiently to accurately represent 
breeding in the outer core zone. When running depletion calculations, the most 
important fission products and $^{233}$Pa are removed while fertile and fissile 
materials are added to the fuel salt every 3 days.  Meanwhile, the removal 
interval for the rare earths, volatile fluorides, and seminoble metals was greater 
than month a (50 days), which caused effective multiplication factor fluctuation. 

\subsection{Equilibrium state search}
The results of this study indicate that the effective multiplication factor 
slowly decreases from 1.075 and reaches 1.02 at equilibrium after approximately 
6 years of operation. At the same time, the concentrations of $^{233}$U, $^{232}$Th, 
$^{233}$Pa, $^{232}$Pa stabilized after approximately 2500 days of operation. 
Particularly, $^{233}$U number density equilibrates\footnote{fluctuates less 
than 0.8\%} after 16 years of operation. Consequently, the core reaches the quasi-equilibrium state after 16 years of operation. However, a wide variety of nuclides, 
including fissile isotopes (e.g. $^{233}$U, $^{239}$Pu) and non-fissile strong 
absorbers (e.g. $^{234}$U), continue accumulating in the core. %The current work results 
%show that a true equilibrium composition cannot exist but balance
%between strong absorber accumulation and new fissile material 
%production can be achieved to keep the reactor critical.

\subsection{Spectral shift}
We also found that the neutron energy spectrum grew harder as the core  
approaches equilibrium because significant heavy fission products accumulated in 
the \gls{MSBR} core. Moreover, the neutron energy spectrum in the central core 
region is much softer than in the outer core region due to lower 
moderator-to-fuel ratio in the outer zone, and this distribution remains stable 
during reactor operation. Finally, the epithermal or thermal spectrum is needed 
to effectively breed $^{233}$U from $^{232}$Th because radiative capture cross 
section of thorium-232 monotonically decreases from $10^{-10}$ MeV to $10^{-5}$ 
MeV. A harder spectrum in the outer core region tends to significantly increase 
resonance absorption in thorium and decrease the absorptions in fissile and 
structural materials. 

The spatial power distribution in the \gls{MSBR} shows that 98\% of the fission 
power is generated in central zone I, and neutron energy spectral shift did not 
cause any notable changes in a power distribution. The spatial distribution of 
neutron capture reaction rate for fertile $^{232}$Th, corresponding to breeding in 
the core, confirms that most of the breeding occurs in an outer, 
undermoderated, region of the \gls{MSBR} core. Finally, the average $^{232}$Th 
refill rate throughout 60 years of operation is approximately 2.40 kg/day or 
100 g/GWh$_e$.

We compared the safety parameters for the initial fuel loading and 
equilibrium compositions using the SERPENT 2 Monte Carlo code. 
The total temperature coefficient 
is large and negative at startup and equilibrium but the magnitude decreases 
throughout reactor operation from $-3.10$ to $-0.94$ pcm/K as the spectrum 
hardens. The moderator 
temperature coefficient is positive and also decreases during fuel depletion. 
The reactivity control system efficiency analysis showed that the safety rod integral 
worth decreases by approximately 16.2\% over 16 years of operation, while 
graphite rod integral worth remains constant. Therefore, neutron energy 
spectrum hardening during fuel salt depletion has an undesirable impact on 
\gls{MSBR} stability and controllability, and should be taken into 
consideration in further analysis of transient accident scenarios.

\subsection{Benefits of fission product removal}

The \gls{MSBR} core performance benefits from the removal of volatile gases, 
noble metals, and rare earths from the fuel salt. 
Moreover, immediate removal of volatile gases (e.g., xenon) and noble metals 
increased reactivity by approximately 7500 pcm over a 10-year 
timeframe. In contrast, the effect of relatively slower removal of rare earth 
elements (every 50 days cycle instead of 3 days) has less impact (5500 pcm) on 
the core reactivity after 10 years of operation. An additional study 
is needed to establish neutronic  and economic tradeoffs of removing each element.

\subsection{Future work}
SaltProc-SERPENT coupled simulation efforts could progress in a 
number of different directions. First optimization of reprocessing parameters (e.g. time step, feeding rate, 
protactinium removal rate) could establish the best fuel utilization, breeding 
ratio, or safety characteristics for various designs. This might be performed with a parameter sweeping 
outer loop which would change an input parameter by a small increment, run the 
simulation and analyze output to determine optimal configuration. Alternatively, 
the existing RAVEN optimization framework \cite{alfonsi_raven_2013} might be 
employed for such optimization studies.

Only the batch-wise online reprocessing approach has been treated in this 
work. However, the SERPENT 2 Monte Carlo code was recently extended for 
continuous online fuel reprocessing simulation \cite{aufiero_extended_2013}. 
This extension must be verified against existing SaltProc/SERPENT or 
ChemTriton/SCALE packages, and could be employed for immediate removal of 
fission product gases (e.g., Xe, Kr) which have a strong negative impact on 
core lifetime and breeding efficiency. Finally, using the built-in SERPENT 2 
Monte Carlo code online reprocessing \& refueling material burnup routine would 
significantly speed up computer-intensive full-core depletion simulations.
