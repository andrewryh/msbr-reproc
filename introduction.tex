\section{Introduction}
% State the objectives of the work and provide an adequate background, avoiding
% a detailed literature survey or a summary of the results.
The \gls{MSR} is an advanced nuclear reactor developed at \gls{ORNL} 
in the 1950s and operated in the 1960s. More recently, the \gls{GIF}  
included \glspl{MSR} among the six most promising advanced reactor concepts for 
further research and development. 
\glspl{MSR} offer significant improvements ``in the four broad areas of 
sustainability, economics, safety and reliability, and proliferation resistance 
and physical protection" \cite{doe_technology_2002}. To achieve the goals 
formulated by the GIF, \glspl{MSR} attempt to simplify the reactor core and 
improve inherent safety by using liquid fuel.

In the thermal spectrum \gls{MSR}, fluorides of fissile and/or fertile 
materials (i.e. UF$_4$, ThF$_4$,  PuF$_3$, TRU\footnote{ Transuranic 
elements}F$_3$) combine with carrier salts to form a liquid fuel that 
circulates in a loop-type primary circuit \cite{haubenreich_experience_1970}. 
Immediate advantages over traditional, solid-fueled, reactors include 
near-atmospheric pressure in the primary loop, 
relatively high coolant temperature, outstanding neutron economy, improved 
safety parameters, reduced fuel preprocessing, and the ability to continuously 
remove fission products and add fissile and/or fertile elements 
\cite{leblanc_molten_2010}. 

The thorium-fueled \gls{MSBR} was developed in the 
early 1970s by \gls{ORNL} specifically to explore the promise of the thorium 
fuel cycle, which uses natural thorium instead of enriched uranium. With 
continuous fuel reprocessing, the \gls{MSBR} realizes the advantages of the 
thorium fuel cycle because the $^{233}$U bred from 
$^{232}$Th is almost instantly\footnote{\space $^{232}$Th transmuted into 
$^{233}$Th after capturing a neutron. Next, this isotope decays to $^{233}$Pa 
($\tau_{1/2}$=21.83m), which finally decays to $^{233}$U 
($\tau_{1/2}$=26.967d).} recycled back into the core 
\cite{betzler_modeling_2016}. The chosen fuel salt, LiF-BeF$_2$-ThF$_4$-UF$_4$, has a 
melting point of $499^\circ$C, a low vapor pressure at operating temperatures, 
and good flow and heat transfer properties \cite{robertson_conceptual_1971}. 
With regard to the nuclear fuel cycle, the thorium cycle produces a reduced quantity 
of plutonium and \glspl{MA} compared to the traditional uranium fuel cycle. 
Finally, the \gls{MSR} also could be employed as a converter reactor for 
transmutation of spent fuel from current \glspl{LWR}.

Liquid-fueled systems present a challenge to existing neutron transport and
depletion tools, which are typically designed to simulate solid-fueled reactors. 
To handle the material flows and potential online removal and feed of 
liquid-fueled systems, early \gls{MSR} simulation methods at \gls{ORNL} 
integrated neutronics and fuel cycle codes (i.e., \gls{ROD} 
\cite{bauman_rod:_1971}) into operational plant tools (i.e., \gls{MRPP}
\cite{kee_mrpp:_1976}) for \gls{MSR} and reprocessing system design.  Based on 
this approach, recent tools from universities and research institutions can
approximate online refueling \cite{serp_molten_2014}. A summary of recent
efforts is listed in table~\ref{tab:fs_codes}.

\begin{table}[ht!]
\caption{Tools and methods for fast spectrum system fuel cycle analysis.}
\begin{tabularx}{\textwidth}{ m | b | x } 
\hline Neutronic code    & \qquad Authors & Spectrum   \\
\hline
\gls{MCNP}/REM \cite{noauthor_mcnp_2004,heuer_simulation_2010}  & Doligez 
\emph{et al.}, 2014; Heuer \emph{et al.}, 2014  
\cite{doligez_coupled_2014,heuer_towards_2014}    & fast \\
\hline
ERANOS \cite{ruggieri_eranos_2006}  & Fiorina \emph{et al.}, 2013 
\cite{fiorina_investigation_2013}            & fast \\
\hline
KENO-IV/ORIGEN \cite{goluoglu_monte_2011,gauld_isotopic_2011}     & Sheu 
\emph{et al.}, 2013 \cite{sheu_depletion_2013} & fast \\
\hline
SERPENT 2 \cite{leppanen_serpent_2015-1}  & Aufiero \emph{et al.}, 2013 
\cite{aufiero_extended_2013} & fast \\
\hline
MCODE/ORIGEN2 \cite{xu_mcode_2008,croff_users_1980}      & Ahmad \emph{et al.}, 
2015 \cite{ahmad_neutronics_2015}   & thermal  \\
\hline
\gls{MCNP}6/CINDER90 \cite{goorley_mcnp6_2013}     & Park \emph{et al.}, 2015; 
Jeong \emph{et al.}, 2016 \cite{park_whole_2015, jeong_equilibrium_2016}& 
thermal\\
\hline
SCALE/TRITON \cite{bowman_scale_2011,powers_new_2013}    & Powers \emph{et al.}, 
2014; Betzler \emph{et al.}, 2017 
\cite{powers_new_2013,powers_inventory_2014,betzler_molten_2017}& thermal\\
\hline
SERPENT 2     & Rykhlevskii \emph{et al.}, 2017 \cite{rykhlevskii_online_2017} & 
thermal\\
\hline
\gls{MCNP}/REM  & Nuttin \emph{et al.} \cite{nuttin_potential_2005}&thermal  \\ 
\hline
\end{tabularx}
  \label{tab:fs_codes}
\end{table}
\FloatBarrier

References 
\cite{doligez_coupled_2014,heuer_towards_2014,sheu_depletion_2013,aufiero_extended_2013} 
simulate some form of reactivity control, and methods 
\cite{doligez_coupled_2014,heuer_towards_2014,aufiero_extended_2013,ahmad_neutronics_2015, 
park_whole_2015, 
jeong_equilibrium_2016,rykhlevskii_online_2017,nuttin_potential_2005} use a set 
of all nuclides in depletion calculations. 

Many liquid-fueled \gls{MSR} designs rely on online fuel processing in which  
material moves to and from the core continuously or at specific time steps 
(batch-wise). In the batch-wise approach, the burn-up simulation stops at a given 
time and restarts with a new liquid fuel composition (after removal of discarded 
materials and addition of fissile/fertile materials). \gls{ORNL} researchers 
have developed ChemTriton, a Python-based script for SCALE/TRITON which uses the 
batch-wise approach to simulate a continuous reprocessing. ChemTriton models salt 
treatment, separations, discharge, and refill using a unit-cell \gls{MSR} 
SCALE/TRITON depletion simulation over small time steps to simulate continuous 
reprocessing and deplete the fuel salt \cite{powers_new_2013}. Methods listed in 
references \cite{fiorina_investigation_2013,sheu_depletion_2013,park_whole_2015, 
jeong_equilibrium_2016,powers_inventory_2014,betzler_molten_2017,rykhlevskii_online_2017} 
as well as the current work also employ a batch-wise approach.

Accounting for continuous removal or addition presents a greater challenge since it 
requires adding a term to the Bateman equations. Aufiero \emph{et al.} explicitly
introduced online fuel reprocessing in the system of equations by adding effective 
decay and transmutation terms for the different nuclides. In his work SERPENT was 
used for solution of the matrix exponential derived from the system of the Bateman 
equations \cite{aufiero_extended_2013}. Similar approach is adopted to model true 
continuous feeds and removals using MCNP transport code listed in references 
\cite{doligez_coupled_2014,heuer_towards_2014,nuttin_potential_2005}.

Thorium-fueled \gls{MSBR}-like reactors similar to the one in this work are 
described in \cite{park_whole_2015, 
jeong_equilibrium_2016,powers_new_2013,powers_inventory_2014, 
betzler_molten_2017,rykhlevskii_online_2017,nuttin_potential_2005}. 
Nevertheless, most of these efforts considered only simplified unit-cell 
geometry because depletion computations for a many-year fuel cycle are 
computationally expensive even for simple models. 

Nuttin \emph{et al.} broke up the reactor core geometry into three \gls{MCNP} cells: 
one for salt channels, one for the salt plena above and below the core, and a 
third cell for the annulus. Consequently, the two-region reactor core was 
approximated by one region with averaged fuel/moderator ratio 
\cite{nuttin_potential_2005}.  Powers \emph{et 
al.}, Betzler \emph{et al.}, and Jeong \emph{et al.} 
\cite{powers_new_2013,powers_inventory_2014,betzler_modeling_2016, 
betzler_molten_2017, jeong_development_2014, jeong_equilibrium_2016} used a 
similar approach. This approach 
misrepresents the two-region breeder reactor concept. The unit-cell or one-region 
models may produce reliable results for homogeneous reactor cores (i.e. 
\gls{MSFR}, \gls{MOSART}) or for one-region single-fluid reactor designs (i.e. 
\gls{MSRE}). However, a two-region \gls{MSBR} must be simulated using a whole-core 
model to capture different neutron transport characteristics in the inner and 
outer regions of the core. In particular, most fissions happen in the inner 
region while breeding occurs in the outer zone.  

Aufiero \emph{et al.} extended the Monte Carlo burnup code SERPENT 2 and 
employed it to study the material isotopic evolution of the 
\gls{MSFR}\cite{aufiero_extended_2013}. The 
developed extension directly accounts for the effects of online fuel 
reprocessing on depletion calculations and features a reactivity control 
algorithm. The extended version of SERPENT 2 was assessed against a dedicated 
version of the deterministic ERANOS-based EQL3D procedure 
\cite{ruggieri_eranos_2006} and adopted to analyze the \gls{MSFR} fuel salt 
isotopic evolution. We employed this extended SERPENT 2 for a simplified 
unit-cell geometry of thermal spectrum thorium-fueled \gls{MSBR} and obtained 
results which contradict existing \gls{MSBR} depletion simulations 
\cite{jeong_equilibrium_2016}.

The present work introduces the online reprocessing simulation package, SaltProc, 
which expands the capability of the continuous-energy Monte Carlo Burnup 
calculation code, SERPENT 2 \cite{leppanen_serpent_2015-1}, for simulation 
liquid-fueled \gls{MSR} operation 
\cite{andrei_rykhlevskii_arfc/saltproc:_2018}. It also reports the 
application of the coupled SaltProc-SERPENT 2 system to the \gls{MSBR}, an 
extension of the work presented in 
\cite{rykhlevskii_full-core_2017, rykhlevskii_online_2017}. In this work, we 
analyzed \gls{MSBR} neutronics and fuel cycle to establish its equilibrium core 
composition. Additionally, we compared predicted operational and safety parameters of the \gls{MSBR} at 
both the initial and equilibrium states to characterize the evolution of its 
safety case over time. Finally, these simulations determined the appropriate $^{232}$Th feed rate  
for maintaining criticality and enabled analysis of the overall \gls{MSBR} fuel 
cycle performance.

The complex \gls{MSBR} geometry is challenging to describe in software input, 
and usually researchers make significant geometric simplifications to model it 
\cite{park_whole_2015}. This study leverages extensive computational 
resources to avoid these geometric approximations in order to accurately capture 
breeding behavior. 
