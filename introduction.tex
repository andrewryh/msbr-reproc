\section{Introduction}
% State the objectives of the work and provide an adequate background, avoiding
% a detailed literature survey or a summary of the results.
The \gls{MSR} is an advanced type of reactor which was developed at \gls{ORNL} in the 1950s and was operated in the 1960s. More recently, \gls{MSR} was included in the six advanced reactor concepts that have been chosen by the Generation IV International Forum (GIF) for further research and development. \glspl{MSR} offer significant improvements ``in the four broad areas of sustainability, economics, safety and reliability, and proliferation resistance and physical protection" \cite{doe_technology_2002}. To achieve the goals formulated by the GIF, \glspl{MSR} attempt to simplify the reactor core and improve inherent safety by using liquid coolant which is also a fuel.

 In the thermal spectrum \gls{MSR}, fluorides of fissile and/or fertile materials (i.e. UF$_4$, ThF$_4$,  PuF$_3$, TRU\footnote{Transuranic elements}F$_3$) are mixed with carrier salts to form a liquid fuel which is circulated in a loop-type primary circuit \cite{haubenreich_experience_1970}. This innovation leads to immediate advantages over traditional, solid-fueled, reactors. These include near-atmospheric pressure in the primary loop, relatively high coolant temperature, outstanding neutron economy, a high level of inherent safety, reduced fuel preprocessing, and the ability to continuously remove fission products and add fissile and/or fertile elements \cite{leblanc_molten_2010}. The thorium-fueled \gls{MSBR} was developed in the early 1970s by \gls{ORNL} specifically to realize the promise of the thorium fuel cycle, which uses natural thorium instead of enriched uranium. With continuous fuel reprocessing, \gls{MSBR} is very attractive to effecively realize advantages of the thorium fuel cycle because the $^{233}$U bred from $^{232}$Th is almost instantly \footnote{$^{232}$Th transmutes into $^{233}$Th after capturing a neutron. Next, this isotope decays to $^{233}$Pa ($\tau_{1/2}$=21.83m), which finally decays to $^{233}$U ($\tau_{1/2}$=26.967d).} being recycled back to the core \cite{betzler_modeling_2016}. The mixture of LiF-BeF$_2$-ThF$_4$-UF$_4$ has a melting point of $499^\circ$C, a low vapor pressure at operating temperatures, and good flow and heat transfer properties \cite{robertson_conceptual_1971}. In the matter of nuclear fuel cycle, the thorium cycle produces a reduced quantity of plutonium and \glspl{MA} compared to the traditional uranium fuel cycle. Finally, the \gls{MSR} also could be employed as a converter reactor for transmutation of spent fuel from current \glspl{LWR}.

Modeling liquid-fueled systems with existing neutron transport and depletion tools is challenging because most of these tools are designed for the solid-fueled reactors simulation. The fuel material flows and potential online separations or feeds of specific elements or nuclides are the main challenges
of liquid-fueled systems. Furthermore, no established tool for liquid-fueled \gls{MSR} neutronics and fuel cycle evaluation exist, though internally developed tools from universities and research institutions can approximate online refueling \cite{serp_molten_2014}. The foundation for these tools was based on early \gls{MSR} simulation methods at \gls{ORNL}, which integrated neutronics and fuel cycle codes (i.e., ROD \cite{bauman_rod:_1971}) into operational plant tools (i.e., MRPP \cite{kee_mrpp:_1976}) for \gls{MSR} and reprocessing system design. More recent research efforts in Europe and Asia mainly focus on fast spectrum reactor fuel cycle analysis and couple external tools to neutron transport and depletion codes take into account continuous feeds and removals in \glspl{MSR}. Four of these efforts are listed in table~\ref{tab:fs_codes}.

\begin{table}[ht!]
\caption{Tools and methods for fast spectrum system fuel cycle analysis.}
\begin{tabularx}{\textwidth}{ x | s | s | m | x } 
\# & Neutronic code  & Depletion code    & \qquad Authors & Spectra   \\
\hline
1 & \gls{MCNP} \cite{noauthor_mcnp_2004}      & REM \cite{heuer_simulation_2010}  & Doligez \emph{et al.}, 2014; Heuer \emph{et al.}, 2014  \cite{doligez_coupled_2014,heuer_towards_2014}    & fast \\
\hline
2 & ERANOS \cite{ruggieri_eranos_2006}      & ERANOS     & Fiorina \emph{et al.}, 2013 \cite{fiorina_investigation_2013}            & fast \\
\hline
3 & KENO-IV \cite{goluoglu_monte_2011}     & ORIGEN \cite{gauld_isotopic_2011}     & Sheu \emph{et al.}, 2013 \cite{sheu_depletion_2013} & fast \\
\hline
4 & SERPENT 2 \cite{leppanen_serpent_2015}   & SERPENT 2  & Aufiero \emph{et al.}, 2013 \cite{aufiero_extended_2013} & fast \\
\hline
5 & MCODE \cite{xu_mcode_2008}      & ORIGEN2 \cite{croff_users_1980}      & Ahmad \emph{et al.}, 2015 \cite{ahmad_neutronics_2015}   & thermal  \\
\hline
6 & \gls{MCNP}6     & CINDER90 \cite{goorley_mcnp6_2013}     & Park \emph{et al.}, 2015; Jeong \emph{et al.}, 2016 \cite{park_whole_2015, jeong_equilibrium_2016}& thermal\\
\hline
7 & SCALE \cite{bowman_scale_2011}      & SCALE/ ChemTriton \cite{powers_new_2013}    & Powers \emph{et al.}, 2014; Betzler \emph{et al.}, 2017 \cite{powers_new_2013,powers_inventory_2014,betzler_molten_2017}& thermal\\
\hline
8 & SERPENT 2      & SERPENT 2     & Rykhlevskii \emph{et al.}, 2017 \cite{rykhlevskii_online_2017} & thermal\\
\hline
9 & \gls{MCNP}      & REM  & Nuttin \emph{et al.} \cite{nuttin_potential_2005}&thermal  \\
\end{tabularx}
  \label{tab:fs_codes}
\end{table}
\FloatBarrier

Most of these methods are also applicable to thermal spectrum \glspl{MSR}. Additional tools developed specifically for thermal \gls{MSR} applications are also listed in table~\ref{tab:fs_codes}.

Methods (1, 3, 4) provide some form of reactivity control, and methods (1, 4, 5, 6, 8, 9) use a set of all nuclides in depletion calculations. 

Liquid-fueled \gls{MSR} designs have online separations and/or feeds, where material is moved to or from the core at all times (continuous) or at specific time steps (batch). To account for batch discharge, a depletion tool must remove some or all material at specified intervals. This requires the burn-up simulation to stop at a given time and restart with a new liquid fuel composition (after removal of discarded materials and addition of fissile/fertile materials). Accounting for a continuous removal or addition is more difficult because it requires adding a term to the Bateman equations. In SCALE \cite{bowman_scale_2011}, ORIGEN \cite{gauld_isotopic_2011} solves a set of Bateman equations using spectrum-averaged fluxes and cross sections generated from a deterministic transport calculation. Methods (1, 4, 8) model true continuous feeds and removals, while other methods employ a batch-wise approach. \gls{ORNL} researchers have developed ChemTriton, a Python-based script for SCALE/TRITON which uses a semi-continuous batch process to simulate a continuous reprocessing. This tool models salt treatment, separations, discharge, and refill using a unit-cell \gls{MSR} SCALE/TRITON model over small time steps to simulate continuous reprocessing and deplete the fuel salt \cite{powers_new_2013}.

Thorium-fueled \gls{MSBR}-like reactors similar to the one in this thesis are described in (6, 7, 8, 9). Nevertheless, most of these efforts considered only simplified unit-cell geometry because depletion computations for a many-year fuel cycle are computationally expesive even for simple models. 

Nuttin \emph{et al.} broke up reactor core geometry into tree \gls{MCNP} cells: one for salt channels, one for two salt plena above and below the core and the last cell for the annulus, consequently, two-region reactor core was approximated by one region with averaged fuel/moderator ratio \cite{nuttin_potential_2005}.  A similar approach was used by Powers \emph{et al.}, Betzler \emph{et al.}, and Jeong \emph{et al.} \cite{powers_new_2013,powers_inventory_2014,betzler_modeling_2016, betzler_molten_2017, jeong_development_2014, jeong_equilibrium_2016} and clearly misrepresent the two-region breeder reactor concept. The unit-cell or one-region models may produce reliable results for homogeneous reactor cores (i.e. \gls{MSFR}, \gls{MOSART}) or for one-region single-fluid reactor designs (i.e. \gls{MSRE}). A two-region \gls{MSBR} must be simulated using a whole-core model to represent different neutron transport in the inner and outer regions of the core, because most fissions happens in the inner region while breeding occurs in the outer zone.  

Aufiero \emph{et al.} extended the Monte Carlo burnup code SERPENT 2 and employed it to study the material isotopic evolution of the \gls{MSFR}. The developed extension directly takes into account the effects of online fuel reprocessing on depletion calculations and features a reactivity control algorithm. The extended version of SERPENT 2 was assessed against a dedicated version of the deterministic ERANOS-based EQL3D procedure \cite{ruggieri_eranos_2006} and adopted to analyze the \gls{MSFR} fuel salt isotopic evolution. We employed this extended SERPENT 2 for a simplified unit-cell geometry of thermal spectrum thorium-fueled \gls{MSBR} and obtained results which contradict existing \gls{MSBR} depletion simulations \cite{jeong_equilibrium_2016}.

The present work introduces the online reprocessing simulation code, SaltProc, which expands the capability of the continuous-energy Monte Carlo Burnup calculation code, SERPENT 2 \cite{leppanen_serpent_2015}, for simulation liquid-fueled \gls{MSR} operation \cite{andrei_rykhlevskii_arfc/saltproc:_2018}. It is also reports the application of the coupled SaltProc-SERPENT 2 system to the \gls{MSBR}, which represents the continuation of the work presented in \cite{rykhlevskii_full-core_2017, rykhlevskii_online_2017}. The major objective of the work herein is to analyze \gls{MSBR} neutronics and fuel cycle to find the equilibrium core composition and core depletion. The additional objective is to compare predicted operational and safety parameters of the \gls{MSBR} at both the initial and equilibrium states. Finally, $^{232}$Th feed rate will be determined and \gls{MSBR} fuel cycle performance will be analyzed.

The \gls{MSBR} complex geometry is hard to describe in software input, and, usually, researchers make significant geometric simplifications to model it \cite{park_whole_2015}. Note that this thesis leverages extensive computational resources to avoid these geometric approximations and accurately capture breeding behavior. This article only discusses liquid-fueled \glspl{MSR}. Discussions of \glspl{MSR} herein refer only to the liquid-fueled variants. Another challenge of the \glspl{MSR}, its delayed neutron precursor drift related to circulating liquid fuel, is not treated here.